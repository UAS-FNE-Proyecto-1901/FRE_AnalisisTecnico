% Márgenes de páginas
\usepackage[
top=2.54cm, 			% Top margin
bottom=2.54cm, 			% Bottom margin
left=3.6cm, 			% Left margin
right=2.54cm,    		% Right margin
headheight=17pt, 		% as per the warning by fancyhdr
includehead,includefoot,
heightrounded, 			% to avoid spurious underfull messages
%showframe, 			% Uncomment to show how the type block is set on the page
]{geometry} 

% Requerido para incluir figuras
\usepackage{graphicx} 

\usepackage{tikz} 
\usetikzlibrary{plotmarks} % Paquete de plotmarks
\usetikzlibrary{shapes,arrows.meta} % Paquete para flechas

% Lenguaje Español, uso de punto en vez e coma para decimales, uso de tabla
\usepackage[spanish,es-nodecimaldot, es-tabla]{babel} 

% Personalizar listas
\usepackage{enumitem} 
\setlist{nolistsep} % Reducir espacio entre viñetas y listas numeradas

% Requerido para líneas horizontales en tablas
\usepackage{booktabs} 
%
\usepackage{multirow}
\usepackage{balance}
\usepackage{float}
\usepackage{amsmath}
\usepackage{ltablex}
\usepackage{tabularx}
\usepackage{afterpage}

% Tablas y Subtablas
\usepackage[labelfont=bf]{caption}
\usepackage[labelformat=simple, labelsep=period]{subcaption}
\renewcommand{\thesubfigure}{\Alph{subfigure}}

% Fuentes
\usepackage{avant}

%----------------------------------------------------------------------------------------
%	Colores definidos por usuario
%----------------------------------------------------------------------------------------
\usepackage{xcolor} % Required for specifying colors by name
% Definición de Color
\definecolor{ocre}{RGB}{243,102,25} 
\definecolor{cyan}{RGB}{0,137,186}
\definecolor{darkergreen}{RGB}{9,147,189}

% Se cambia por el color Pantone del año 2020 Azul Clásico
\definecolor{specialgreen}{RGB}{15,76,129} % Azul Clásico Pantone 19-4052
\definecolor{babyblue}{RGB}{181,199,211} % Baby Blue Pantone 13-4308 TCX
\definecolor{cornhusk}{RGB}{243,213,173} % Cornhusk Pantone 12-0714 TCX
\definecolor{Peach-Quartz}{RGB}{245,184,149} % Peach Quartz Pantone 13-1125 TCX

% Colores de Portada
\definecolor{bgBlue}{RGB}{42,43,53}
\definecolor{txBlue}{RGB}{80,179,209} 
\usepackage{mathptmx} 

%----------------------------------------------------------------------------------------
%	Bibliografía e índices
%----------------------------------------------------------------------------------------
%\usepackage{csquotes}	\usepackage[style=numeric,citestyle=numeric,sorting=nyt,sortcites=true,autopunct=true,
%autolang=hyphen,hyperref=true,abbreviate=false,backref=true,backend=biber,defernumbers=true]{biblatex}
%\addbibresource{bibliography.bib} % BibTeX bibliography file
%\defbibheading{bibempty}{}

% Paquete de Referenciación Natbib
\usepackage[numbers, round, sort, sectionbib]{natbib}
\renewcommand{\bibsection}{} % Para quitar la palabra bibliografía como título de esta sección o colocar otra. 
\setcitestyle{square}
\bibliographystyle{vancouver}

% Para un cálculo simple: se utiliza para espaciar correctamente los encabezados de las 
% letras índice
\usepackage{calc} 
% Requerido para hacer un índice
\usepackage{makeidx} 
% que Crear los archivos necesarios para la indexación en Latex
\makeindex 

%----------------------------------------------------------------------------------------
%	Encabezado de página
%----------------------------------------------------------------------------------------
% Requerido para la configuración de encabezado y pie de página
\usepackage{fancyhdr} 

\pagestyle{fancy}
% Configuración de fuente de texto de capítulo
\renewcommand{\chaptermark}[1]{\markboth{\sffamily\normalsize\bfseries\chaptername\ \thechapter.\ #1}{}} 
% Configuración de fuente de texto de sección
\renewcommand{\sectionmark}[1]{\markright{\sffamily\normalsize\thesection\hspace{5pt}#1}{}} 
% Configuración de fuente para el número de página en el encabezado
\fancyhf{} \fancyhead[LE,RO]{\sffamily\normalsize\thepage} 
\fancyhead[LO]{\rightmark} % Imprima el nombre de la sección más cercana en el lado izquierdo de las páginas impares
\fancyhead[RE]{\leftmark} % Imprima el nombre del capítulo actual en el lado derecho de las páginas pares
\renewcommand{\headrulewidth}{0.5pt} % De ancho de la regla debajo del encabezado
\addtolength{\headheight}{2.5pt} % Aumente ligeramente el espacio alrededor del encabezado
\renewcommand{\footrulewidth}{0pt} % Elimina la regla en el pie de página
\fancypagestyle{plain}{\fancyhead{}\renewcommand{\headrulewidth}{0pt}} % Estilo para cuando se especifica un estilo de página simple

%% Elimina el encabezado de páginas vacías impares al final de los capítulos
%\makeatletter
%\renewcommand{\cleardoublepage}{
%	\clearpage\ifodd\c@page\else
%	\hbox{}
%	\vspace*{\fill}
%	\thispagestyle{empty}
%	\newpage
%	\fi}


%% Esto se usa para prevenir el error de flotas sobrepasados en ciclos
\usepackage[maxfloats=256]{morefloats}
